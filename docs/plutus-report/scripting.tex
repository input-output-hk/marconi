\section{Scripting: Plutus Core}
\label{sec:plutus-core}

Our scripting language for the \gls{plutus-platform} is \gls{plutus-core} (strictly, type-erased \gls{plutus-core}, see \cref{sec:erasure}).
A formal specification for \gls{plutus-core} is available in \textcite{plutus-core-spec}.

We give a high-level overview of the language here, further details can be found in the above reference.

\subsection{Requirements}
\begin{requirement}[Conservatism]
\label{req:script-lang-conservatism}
Designing a new programming language is hard.
It is very easy to make choices that come back to haunt you.

Moreover, whatever language we choose for our scripting language will be very hard to change, since it will be involved in transaction validation, and we want to be able to validate old transactions.
So we can release new versions, but we must support old versions forever.

This combination means that it is hard for us to get the language right first time, and it is also hard for us to iterate on it.
This suggests a case for \emph{conservatism}: pick things that are tried and tested, and don't try to innovate too much.

Conservatism also makes requirements such as \cref{req:script-lang-formalization} easier to satisfy, as we can build on existing work.
\end{requirement}

\begin{requirement}[Minimalism]
\label{req:script-lang-minimalism}
The smaller our language, the less there is to go wrong, and the less there is to reason about.

Minimalism makes requirements such as \cref{req:script-lang-formalization} easier to satisfy, since there is less to formalize.

However, this is a tradeoff, as a simpler target language often means more complexity in the compilers that target that language.
But it is much easier for us to change the compilers than it is to change the scripting language, so this is worth it.
\end{requirement}

\begin{requirement}[Safety]
\label{req:script-lang-reasoning}
Once submitted as part of a blockchain transaction, scripts are immutable.
Therefore we want as much certainty as we can get about what the code will do, otherwise there is risk that the value involved will be lost or stolen.
\end{requirement}

\begin{requirement}[Formalization]
\label{req:script-lang-formalization}
One part of reasoning about what our programming language does is to \emph{formalize} its semantics, so we can be sure that
\begin{inparaenum}
  \item it has a sensible semantics, and
  \item the implementation agrees with that semantics
\end{inparaenum}.
\end{requirement}

\begin{requirement}[Size]
\label{req:script-lang-size}
The representation of the scripting language on the chain must not be too large, since
\begin{inparaenum}
\item users will pay for the size of transactions, and
\item transaction size has a major effect on the throughput of the system.
\end{inparaenum}
\end{requirement}

\begin{requirement}[Extensibility]
\label{req:script-lang-extensibility}
We may want to use our scripting language on other ledgers in the future, and they may well support different basic primitives, so we need this to be configurable.
For example, privacy-preserving ledgers may restrict the kinds of computation that can be done, which may mean we only have some kinds of arithmetic operations.
\end{requirement}

\begin{requirement}[Multiple source languages]
\label{req:source-lang-multiple}
It would be nice if multiple source languages could be compiled to our scripting language.
This would potentially encourage usage by developers who are more comfortable with one or the other of the source languages.

However, this is somewhat speculative and actually implementing other source languages would be a lot of work.
But if we can cheaply make this easier then that is good.
\end{requirement}

\subsection{Designing Plutus Core}

How do we design our scripting language?
\Cref{req:script-lang-conservatism,req:script-lang-minimalism} suggest adhering to existing languages as much as possible, and picking a small, well-studied language.
We also want to use a statically-typed functional programming language, since we intend to use functional programming languages (starting with Haskell) as our source languages.

This suggests using some variant of the lambda calculus.
We decided to start with \emph{\gls{system-f}}, also known as the polymorphic lambda calculus \autocite{Girard-thesis}.
Haskell's internal language, \gls{ghc-core} \parencite{jones1998transformation}, is also based on \gls{system-f} (no coincidence), but we make a couple of different decisions:
\begin{enumerate}
  \item We do not have primitive datatypes and case expressions, rather we base our language on \emph{\gls{system-fomf}}, which extends \gls{system-f} with recursive types and higher-kinded types.
  \item Our language is strict, rather than lazy, by default.
  \item We do not support most of the extensions that Haskell has pioneered, such as coercions.
\end{enumerate}

As a result, the formal specification of our language can be described in one line: it is exactly \gls{system-fomf} with appropriate primitive types and operations.


\subsection{Datatypes}
If we do not have primitive datatypes, how \emph{do} we deal with datatypes?
The answer is that it is up to the compiler to encode them --- another example of the tradeoff discussed in \cref{req:script-lang-minimalism}.

In our case \gls{plutus-ir} does have datatypes, so this is handled by the \gls{plutus-ir} compiler. See \cref{sec:plutus-ir} for more details.

\subsection{Recursive types}

The one-line description above turns out not to be as unambiguous as one might hope. We have
to choose between equirecursive types and isorecursive types \autocite[chapter 21]{pierce2002types}.

There is a tradeoff here between simplicity of writing code in the language, and simplicity of the language's metatheory.
Since \gls{plutus-core} is a compilation target rather than a source language, we opted to go for isorecursive types, which have the simpler metatheory (aiding \cref{req:script-lang-formalization}).
The complexity is handled by the \gls{plutus-ir} compiler.

This choice is discussed more in \textcite{plutus-core-spec, peytonjones2019unraveling}.

\subsection{Recursive values}

While \gls{plutus-core} has support for recursive types, it does not have any (direct) support for recursive \emph{values}.
It turns out that recursive types are sufficient to implement the usual array of fixpoint combinators, and so encode recursive values \autocite{harper2012practical}.
Again, this encoding is handled by the \gls{plutus-ir} compiler, see \cref{sec:plutus-ir}.

Doing this in full generality turns out to be surprisingly tricky, see \textcite{peytonjones2019unraveling}.

\subsection{Builtin types and values}

Programming languages commonly have some form of foreign function interface. The idea is that:

\begin{enumerate}
\item
  We don't want to reimplement everything from the bottom up in a newly designed language.
  It makes much more sense to delegate certain computations to an existing tool where such computations already exist in optimized form, with good tests etc.
\item
  In particular, We certainly don't want to implement our own version of arithmetic (integers, addition, multiplication, etc.) or any other low-level stuff where it will be especially hard to beat external implementations.
\end{enumerate}

We certainly need efficient primitives in \gls{plutus-core}, including arithmetic.

To implement this, we could build \code{Integer} into the grammar of \gls{plutus-core} types, and \code{Integer} constants into the grammar of \gls{plutus-core} terms, as well as a few functions (addition, multiplication, etc.).
If we need more types or functions, then we simply have to extend the hardcoded list of types and functions.

However, this is very inflexible.
The key issue is that we may want to have \emph{different} sets of builtins in different situations.
For example:
\begin{enumerate}
\item
  During testing it can be convenient to have a builtin which implements \emph{tracing}: emitting log lines to a global trace output.
  But we don't want this functionality on-chain, so we want to remove the builtin (or replace it with a no-op).
\item
  We may need different builtins for different ledgers (see \cref{req:script-lang-extensibility}).
\end{enumerate}

So we need to make builtin types and functions extensible.

\subsubsection{Builtin types}

Extensible builtin types are relatively straightforward on the type level where they are opaque --- we don't need to know anything about a type other than its kind in order to handle type normalization etc.
However, values of builtin types (``constants'' in \gls{plutus-core}) are trickier.
It's easy to say what a \code{Integer} constant should be: an integer!
However, what about constants of extensible builtin types?

We adopt an approach familiar from generic programming in dependently-typed languages, where we use a \emph{universe} of types to represent the types our programs can operate over.\footnote{
  Where do these types themselves come from?
  We assume that we are working in some ``metalanguage'' which has types such as \code{Integer} that we can reuse in Plutus Core.
}
In our case, we are entirely parametric over the universe itself, with different universes corresponding to different sets of builtin types.
A constant then is then \emph{tagged} with a value corresponding to a type in that universe, and a value of that type.

This does complicate interpretation of \gls{plutus-core} programs, since in order to interpret a program, you also need to know what universe of types it is expecting.
In practice this is not so bad: for most users (e.g. the \gls{cardano} ledger) there will be a single universe that they use, and from there on they can simply act as though we did not have extensible builtins.

\paragraph{Builtin datatypes}

Adding new types to \gls{plutus-core} is not very useful unless they have functions that can operate on them!
A particularly common case is where the new type is a datatype, and should support pattern matching.
We can support this by also providing a ``matching'' function as a builtin function.\footnote{
  These functions look just like the ``matchers'' for Scott-encoded datatypes, which is no coincidence.
}

For example, the matching function for \code{bool} is \code{ifThenElse :: forall a . bool -> a -> a -> a}.
Matching functions are always polymorphic, which is a key reason why we need to support polymorphic builtin functions (see \cref{sec:polymorphic-builtins}).

\paragraph{Polymorphic builtin types}

It would be nice to support polymorphic builtin types.
For example, that would allow us to provide optimized implementations of container types such as maps.

This seems feasible in principle, but we have not worked the details out yet.

\subsubsection{Builtin functions}

We could implement extensible builtin functions in a similar way to extensible builtin types: use some type (like the universe for types) that catalogues the available builtin functions.
However, in practice we don't need to model the available functions quite as precisely, since the syntax doesn't depend on the available functions as it does on the available types.

So instead we adopt a much simpler approach where the available builtin functions can be provided at runtime simply indexed by a name.
We may change this in future to use the explicit catalogue of functions.

\paragraph{Passing arguments to builtin functions}

Builtin functions must of course be provided with a type signature (since we have a typed language), and a way to evaluate them.
The natural way to provide such evaluators is as functions --- but these functions must be able to operate on \gls{plutus-core} terms!
For example, an evaluator for ``plus'' must be able to take two \emph{terms} of type \code{Integer} and turn them into another term.
But a simple implementation of ``plus'' just talks about \emph{integers}, not terms.

Solving this problem in generality is difficult, so we make a key simplification: builtin functions may only accept arguments of builtin types (with one exception, as we will see shortly).
Along with the fact that (since \gls{plutus-core} is eager) function arguments are evaluated to values before being passed to the function, this means that the arguments to builtin evaluators will always be \emph{constants}, which are easy to unwrap into their underlying values.

\paragraph{Polymorphic builtin functions}
\label{sec:polymorphic-builtins}

The exception to the above rule about builtin function types is for \emph{polymorphic} builtin functions.
These must necessarily be able to process arbitrary terms (since they can be instantiated at any type, not just a builtin type).
For example, \code{ifThenElse} must be able to produce a value of type \code{Integer -> Integer}, even though this is not a builtin type.

It is possible in principle to use polymorphic \emph{evaluators} to implement polymorphic builtin functions \autocite{lindley2012embedding}.
Unfortunately this is quite a heavyweight and complex encoding for such a key usecase.

Instead, we require evaluators for polymorphic functions to be \emph{parametric}, which means in particular that they do not inspect their polymorphic arguments.
We can thus pass the \emph{term itself} as the polymorphic argument, instead of needing to extract a value of an appropriate type in the metalanguage, as we do for other arguments.
Thus, an evaluator for a builtin function which implements \code{const :: forall a b . a -> b -> a} would take two \emph{terms} and return the first one.

\paragraph{Saturated builtin application}

When including builtin function application in the AST, we have two options:
\begin{enumerate}
\item
  Builtin functions can stand alone as a value with the builtin's function type, and are applied to their arguments with normal function application.
\item
  Builtin functions must always appear \emph{saturated}, that is, with all of their arguments.
\end{enumerate}

There choice is fairly minor, but has a few consequences:
\begin{enumerate}
\item
  Unsaturated builtins can be partially applied, saturated builtins cannot be --- but we can easily work around this by wrapping the builtin in a lambda, so this does not matter in practice.
\item
  Saturated builtins are easier to write an efficient evaluator for.
  Unlike normal functions, we cannot do any real evaluation of a builtin function until we have \emph{all} the arguments.
  So it is easier if we only ever see them with all their arguments.
\end{enumerate}

We've chosen to use saturated builtin applications.

\paragraph{Errors from builtin functions}

Builtin functions \emph{are} currently allowed to fail, i.e. to use the \code{error} effect in \gls{plutus-core}.
This allows us to handle cases like $1 \div 0$ without having to complicate the builtin machinery with an explicit error handling facility like \code{Maybe}.

Builtin functions are \emph{not} currently allowed to catch errors (nothing in \gls{plutus-core} can catch an error, currently).

\subsection{Erasure}
\label{sec:erasure}

Originally we planned to use typed \gls{plutus-core} as the actual scripting language.
However, we discovered that the explicit types made up a large proportion of the overall size of the code ($\approx 80\%$).
Given that we care about size (\cref{req:script-lang-size}), this was too compelling an improvement to pass by.

Hence we decided to instead use \emph{type-erased} \gls{plutus-core} as our scripting language instead, with typed \gls{plutus-core} as a (useful) intermediary language.

We were initially concerned that if we did not typecheck the application of the \gls{validator} to its arguments (which we cannot do if we've erased the types of the \gls{validator}!), then that might allow a malicious attacker to pass a script ill-typed arguments, potentially causing unexpected behaviour.
However, given the current design where all the arguments to the \gls{validator} are of type \gls{data}, and are constructed by the validating node (which is a trusted party), it is no longer possible for the arguments to be ill-typed.

Erased \gls{plutus-core} is much closer to the untyped lambda calculus, and as such is also an easier compilation target (e.g. for a dependently-typed language, or a non-statically-typed language), hence this also helps with \cref{req:source-lang-multiple}.

\subsection{Formalization}

As discussed in \cref{req:script-lang-formalization}, we would like to formalize \gls{plutus-core}.
We have done so in Agda: the resulting formalization is partially published in \textcite{chapman2019system}, and the living version is available in \textcite{plutus-repo}.

Our formalization includes the type system and semantics, proofs of progress and preservation, and an evaluator which provably implements the semantics.
This evaluator can be extracted to a Haskell executable, which we use to cross-test against the Haskell interpreter.

\subsection{Costing}
\label{sec:costing}

The \gls{plutus-core} evaluator can track the \emph{cost} of evaluating a program.
The \gls{plutus-core} evaluator can return the amount of resources consumed while evaluating a program, or work within a preset resource limit, halting if it is exceeded.
The ledger itself is responsible for deciding how many \gls{exunits} the evaluation of a particular \gls{script} should be allowed.

\subsubsection{Exunits}

Resources are tracked in terms of two abstract units: abstract time, and abstract (peak) memory.
Together we refer to these as \gls{exunits}.

Why are the units here ``abstract''?
They must be, because we need to be able to keep things deterministic (\cref{req:ledger-determinism}), and so we cannot use \emph{real} time or memory usage.
Rather we have to define some abstract measures which we try to align with real resource usage.

\subsubsection{Cost model}

The \gls{plutus-core} evaluator is parameterized by a \gls{cost-model}.
This is a set of parameters which affects how the evaluator assigns resource costs to various operations during evaluation.
For example, there might be a parameter setting a multiplicative parameter for the time taken to do an integer addition.
Eventually these will be stabilised and formalised as part of \textcite{plutus-core-spec}.

We need to carefully calibrate the cost model so that the abstract costs we assign are well correlated with real costs.
In combination with appropriate prices for \gls{exunits} this should ensure that script evaluation is economically viable for \glspl{slot-leader}.
