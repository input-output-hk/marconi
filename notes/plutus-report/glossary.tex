% General
\newglossaryentry{ada}{
  name=ada,
  description={
    The primary currency on the \gls{cardano} blockchain. Unlike other \glspl{currency}, ada cannot be forged.
  }
}
\newglossaryentry{cardano}
{
  name=Cardano,
  description={A third-generation blockchain developed by IOHK.}
}
\newglossaryentry{daedalus}
{
  name=Daedalus,
  description={The primary wallet frontend developed by IOHK for use with \gls{cardano}.}
}
\newglossaryentry{plutus-platform}
{
  name=Plutus Platform,
  description={
    The combined software support for writing \glspl{app}.
  }
}
\newglossaryentry{slot-leader}
{
  name=slot leader,
  description={
    The servers adding new blocks to the chain in the proof-of-stake Ouroboros protocol, corresponding to the miners of a proof-of-work protocol as employed in Bitcoin.
  }
}
\newglossaryentry{node}
{
  name=node,
  description={
    The user's local interface to the blockchain.
  }
}
\newglossaryentry{wallet-backend}
{
  name=wallet backend,
  description={
    The service that provides most of the wallet functionality, e.g. balance tracking, transaction submission, key management.
  }
}
\newglossaryentry{wallet-frontend}
{
  name=wallet frontend,
  description={
    A graphical user interface to a wallet, typically backed by a \gls{wallet-backend}, e.g. \gls{daedalus}.
  }
}

% Ledger
\newglossaryentry{address}{
  name=address,
  description={
    The address of an \gls{utxo} says where the output is ``going''.
    The address stipulates the conditions for unlocking the output.
    This can be a public key hash, or (in the \gls{eutxo-model}) a \gls{script} hash.
  }
}
\newglossaryentry{context}{
  name=context,
  description={
    A data structure containing a summary of the transaction being validated.
    See \cref{sec:eutxo}.
  }
}
\newglossaryentry{currency}{
  name=currency,
  plural=currencies,
  description={
    A class of \gls{token} whose \gls{forging} is controlled by a particular \gls{mps}.
  }
}
\newglossaryentry{currency-id}{
  name=currency id,
  description={
    An identifier for a \gls{currency}.
    This is the hash of the \gls{mps} that controls the \gls{forging} of the \gls{currency}.
  }
}
\newglossaryentry{data}{
  name=\textsf{Data},
  description={
    A generic type of structured data.
    See \cref{sec:data}.
  }
}
\newglossaryentry{datum}{
  name=datum,
  description={
    The data field on script outputs in the \gls{eutxo-model}.
    See \cref{sec:eutxo}.
  }
}
\newacronym[description={Our extended ledger model. See \cref{sec:eutxo}.}]{eutxo}{EUTXO}{Extended UTXO}
\newglossaryentry{eutxo-model}
{
  name=Extended UTXO Model,
  description={
    The ledger model which the \gls{plutus-platform} relies on.
    This is implemented in the Goguen release of \gls{cardano}.
    See \cref{sec:eutxo}.
  }
}
\newglossaryentry{forging}{
  name=forging,
  description={
    A transaction which forges \glspl{token} creates new \glspl{token}, providing that the
    corresponding \gls{mps} is satisfied.
    The amount forged can be negative, in which case the \glspl{token} will be destroyed instead of created.
    See \cref{sec:multicurrency}.
  }
}
\newglossaryentry{fungible}
{
  name=fungible,
  description={
    A \gls{token} is \emph{fungible} with another token if it can be used interchangeably with that token.
  }
}
\newglossaryentry{ledger}{
  name=ledger,
  description={
    A system for tracking ownership and transfers of assets.
    We will usually use this to mean ``distributed ledger''.
    See \cref{sec:ledger}.
  }
}
\newacronym[description={
  A \gls{script} which must be satisfied in order for a transaction to forge \glspl{token} of the corresponding \gls{currency}.
  See \cref{sec:multicurrency}.
}]{mps}{MPS}{Monetary Policy Script}
\newglossaryentry{multicurrency}
{
  name=multicurrency,
  description={
    A generic term for a ledger which supports multiple different currencies natively.
    See \cref{sec:multicurrency}.
  }
}
\newacronym[description={
  A unique token which is not \gls{fungible} with any other token.
}]{nft}{NFT}{Non-Fungible Token}
\newglossaryentry{one-shot}
{
  name=one-shot,
  description={
    A one-shot script is a script that requires that the pending transaction spend a specific existing transaction output.
    Since transaction outputs can only be spent once, this ensures that the script can only be (successfully) run once also.
    We may also refer to a one-shot transaction, which is a transaction that contains a one-shot script.
  }
}
\newglossaryentry{redeemer}
{
  name=redeemer,
  description={
    The argument to the validator \gls{script} which is provided by the transaction which spends a \gls{script-output}.
    See \cref{sec:eutxo}.
  }
}
\newglossaryentry{script}
{
  name=script,
  description={
    A generic term for an executable program used in the ledger.
    In the \gls{plutus-platform}, these are written in \gls{plutus-core}.
    See \cref{sec:eutxo}.
  }
}
\newglossaryentry{script-output}
{
  name=script output,
  description={
    A transaction output locked by a \gls{script}.
    See \cref{sec:eutxo}.
  }
}
\newglossaryentry{token}
{
  name=token,
  description={
    A generic term for a native tradeable asset in the ledger.
  }
}
\newglossaryentry{token-name}{
  name=token name,
  description={
    An identifier for a class of \glspl{token} within a \gls{currency}.
  }
}
\newacronym[description={
  A transaction output which has not been spent.
  Also refers to the traditional ledger model going back to Bitcoin.
}]{utxo}{UTXO}{Unspent Transaction Output}
\newglossaryentry{validator}
{
  name=validator script,
  description={
    The \gls{script} attached to a \gls{script-output} in the \gls{eutxo-model}.
    Must be run and return positively in order for the output to be spent.
    Determines the \gls{address} of the output.
    See \cref{sec:eutxo}.
  }
}
\newglossaryentry{value}
{
  name=\textsf{Value},
  description={
    A type representing a bundle of assets (\glspl{token}) in our \gls{multicurrency} system.
    See \cref{sec:value}.
  }
}

% Scripting
\newglossaryentry{cost-model}
{
  name=cost-model,
  description={
    A set of parameters which affect how the \gls{plutus-core} evaluator costs a program evaluation.
    See \cref{sec:costing}.
  }
}
\newglossaryentry{exunits}
{
  name=exunits,
  description={
    The units used to track execution costs.
    A pair of ``abstract time'' and ``abstract memory''.
    See \cref{sec:costing}.
  }
}
\newglossaryentry{space}
{
  name=Space,
  description={
    The abstract unit of memory.
    See \cref{sec:costing-units}.
  }
}
\newglossaryentry{time}
{
  name=Time,
  description={
    The abstract unit of time.
    See \cref{sec:costing-units}.
  }
}
\newglossaryentry{plutus-core}
{
  name=Plutus Core,
  description={
    The programming language used for \glspl{script} in the \gls{plutus-platform}.
    See \cref{sec:plutus-core}.
  }
}
\newglossaryentry{system-f}
{
  name={\ensuremath{\textrm{System}\ F}},
  description={
    A well-known simple programming language, often called the ``polymorphic lambda calculus''.
  }
}
\newglossaryentry{system-fomf}
{
  name={\ensuremath{\textrm{System}\ F_{\omega}^\mu}},
  description={
    An extension of \gls{system-f} with recursive types and higher-kinded types.
  }
}

% Applications
\newglossaryentry{app}
{
  name=Plutus application,
  description={
    TODO: better name?
    An application built using the \gls{plutus-platform}.
  }
}
\newglossaryentry{app-api}
{
  name=contract API,
  description={
    TODO: better name?
    The interface provided by the \gls{paf} which \glspl{app} must use.
  }
}
\newglossaryentry{app-inst}
{
  name=application instance,
  description={
    A particular instance of an \gls{app}.
  }
}
\newglossaryentry{app-exe}
{
  name=application executable,
  description={
    The compiled executable from an \gls{app}.
  }
}
\newglossaryentry{chain-index}
{
  name=chain index,
  description={
    A component of the \gls{pab} that tracks information from the chain, notably output \glspl{datum}.
  }
}
\newglossaryentry{marlowe}{
  name=Marlowe,
  description={A domain-specific language for writing financial contract applications.}
}
\newglossaryentry{off-chain}
{
  name=off-chain code,
  description={
    The part of an \gls{app}'s code which runs off the chain, usually on a user's computer.
  }
}
\newglossaryentry{on-chain}
{
  name=on-chain code,
  description={
    The part of an \gls{app}'s code which runs on the chain (i.e. as scripts).
  }
}
\newacronym[description={
  The component which manages \glspl{app} run on users' machines.
  See \cref{sec:pab}.
}]{pab}{PAB}{Plutus Application Backend}
\newacronym[description={
  The overall framework for writing and running \glspl{app}.
  See \cref{sec:paf}.
}]{paf}{PAF}{Plutus Application Framework}

% Authoring
\newglossaryentry{ghc-core}
{
  name=GHC Core,
  description={
    GHC's internal representation of Haskell.
    A variant of \gls{system-f}.
  }
}
\newglossaryentry{plutus-ir}
{
  name=Plutus IR,
  description={
    An intermediate language that compiles to \gls{plutus-core}.
    See \cref{sec:plutus-ir}.
  }
}
\newglossaryentry{plutus-playground}
{
  name=Plutus Playground,
  description={
    A web based environment for trying out the \gls{plutus-platform}.
    See \cref{sec:plutus-playground}.
  }
}
\newglossaryentry{plutus-sdk}
{
  name=Plutus Haskell SDK,
  description={
    The libraries and development tooling for writing \glspl{app} in Haskell.
    See \cref{sec:sdk}.
  }
}
\newglossaryentry{plutus-tx}
{
  name=Plutus Tx,
  description={
    The libraries and compiler for compiling Haskell into \gls{plutus-core} to form the on-chain part of an \gls{app}.
    See \cref{sec:plutus-tx}.
  }
}
