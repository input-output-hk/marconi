\section{The ledger: the Extended UTXO Model}
\label{sec:ledger}
\todompj{Maybe we should just move the content of the EUTXO doc into here?}

The \gls{plutus-platform} is designed to work with a specific kind of ledger, namely an extended form of the traditional \gls{utxo} ledger introduced by Bitcoin.
We refer to our ledger model as the \gls{eutxo} Model, and it is discussed in detail in \textcite{eutxo,utxoma,eutxoma}.

We give a high-level overview here, further details can be found in the above references.

\subsection{Requirements}
\begin{requirement}[Locality]
\label{req:ledger-locality}
\Gls{utxo} ledgers have the desirable property that all the information which is needed to validate a transaction is specified in the transaction.
In particular, there is no global state apart from the \gls{utxo} set.

This is in contrast to account-based ledgers, where account balances are global.\footnote{
  In fact Cardano has a small maount
}

We aim not to disturb this property, since it is helpful in a number of ways.
For example, it allows a greater degree of parallel processing.
In \textcite{chakravartyhydra} this property is used to enable fast optimistic settlement of transactions off-chain, \emph{including} support for the scripting model we use.
This would not be possible if we followed Ethereum's actor model and had contract instances with global state.
\end{requirement}

\begin{requirement}[Determinism]
\label{req:ledger-determinism}
We would like our ledger rules to be \emph{deterministic}, that is, the outcome of transaction validation is entirely determined by the transaction in question (and the current state of the ledger of course).
This means that the whole process can be simulated accurately by the user \emph{before} submitting a transaction: thus both the outcome of validation and the amount of resources consumed can be determined ahead of time.
This is helpful for systems that charge for \gls{script} execution, since users can reliably compute how much they will need to pay ahead of time.

A common way for systems to violate this property is by providing \glspl{script} with access to some piece of mutable information, such as the current time (in our system, the current slot has this role).
\Glspl{script} can then branch on this information, leading to non-deterministic behaviour.
\end{requirement}

\begin{requirement}[Inspectability]
\label{req:ledger-inspectability}
As far as possible, we would like the data which we put on the ledger to be \emph{inspectable}.
Hence the \gls{off-chain} needs to be able to inspect the \gls{datum} and interpret it as a value it can work with.

This is primarily important for \gls{off-chain}.
For example, when implementing \glspl{app} which are state machines, we will use the \glspl{datum} of outputs to store the on-chain \emph{state} of the machine.
If the \gls{off-chain} wants to submit a transaction to advance such a state machine, it may need to read the current state from the chain (after all, the last transition might not have been performed by this agent).
\end{requirement}

\begin{requirement}[UTXO size]
\label{req:ledger-utxo-size}
\Glspl{slot-leader} are likely to be faced with resource constraints when doing transaction validation.
A key source of memory usage is that the whole set of current \gls{utxo} entries must be kept in memory at all times to check for double-spending.
Hence, it is important to keep the size of \gls{utxo} entries small.
\end{requirement}

\subsection{The \glsentryname{data} type}
\label{sec:data}

We will later require the ability to pass arguments to \glspl{script}.
However, this requires us to turn such arguments into values in our scripting language of the appropriate type.

We have opted to use a single, structured datatype for interchange with \glspl{script}.
We call this type \gls{data}, and it is described further in \textcite{eutxo}.\footnote{
  In practice it is very similar to other forms of structured data, such as JSON, but optimized to work with CBOR \autocite{cbor}.
}
This means that the \gls{redeemer} and \gls{datum} are of type \gls{data}, and the \gls{context} is encoded as \gls{data} before being passed to the \gls{validator}.

We also require a way to turn the \gls{data} values into values in our scripting language, but we can do this via the lifting machinery in \cref{sec:plutus-tx-lifting}.

On the user side, users are responsible for turning their specific types into \gls{data}, but this can be done in a standardized way, and we provide Template Haskell functions for generating the appropriate typeclass instances.

\subsection{UTXO ledgers}

A \gls{utxo} ledger represents a ledger as a series of \emph{transactions}, each of which has a set of transaction \emph{inputs} and transaction \emph{outputs}.

A transaction output carries some amount of \emph{value} and is \emph{locked} in some way that restricts how it can be spent.
The simplest form of locking is simply to attach a public key.

A transaction input is a reference to a transaction output.
For example, a pair of a transaction ID and an index into the (ordered) set of its outputs.

\subsubsection{UTXO validity}
\label{sec:utxo-valid}

For a transaction to be \emph{valid} against an existing \gls{utxo} ledger, it must be the case that:
\begin{itemize}
\item All inputs refer to outputs which exist in the ledger and have not been spent by another transaction in the ledger.
\item The transaction balances: the sum of the value on the spent outputs equals the sum of the value on the transaction outputs.
\item The transaction is authorized: for each spent output locked by a public key, the transaction contains a witness (a signature over the transaction body) from the corresponding private key.
\end{itemize}
\medskip

Hence ownership in a \gls{utxo} ledger consists in having the right to spend an output: one owns outputs rather than an aggregate amount of funds.

\subsection{EUTXO-1: scripting}
\label{sec:eutxo}

\todompj{This section is a mess and doesn't tell the story well.}

The first addition we make is to support scripting.
Our approach is based of existing models for \gls{utxo} ledgers with scripts (e.g. \textcite{Zahnentferner18-UTxO}), but we make a number of additions.

We make the following changes to the \gls{utxo} model:

\begin{itemize}
\item
  Every transaction has a \emph{validity interval} over slots.
  A \gls{slot-leader} will only process the transaction if the current slot number lies within the transaction's validity interval.

\item
  The ledger learns about a \emph{scripting language} and how to evaluate scripts in that language.\footnote{
    In the \gls{plutus-platform} this language is \gls{plutus-core}.
    }

\item
  Transaction outputs can be locked by a \gls{validator} (such outputs are called \glspl{script-output}).

\item
  \Glspl{script-output} have an attached \gls{datum}, which is a value of type \gls{data}.

\item
  When spending a \gls{script-output}, the spending transaction must provide a \gls{redeemer} of type \gls{data} for each such output.

\item
  Validator scripts are passed three arguments: the \gls{datum} of the output being spent, the \gls{redeemer} for the output, and a value called the \gls{context} which has a representation of the transaction being validated.
\end{itemize}

\todompj{I'm totally glossing over the details about which things are hashes and the whole data witnesses thing.}

\subsubsection{The contents of the context}
The \gls{context} type is a summary of the information contained in a transaction, made suitable for consumption in a \gls{script}.

The exact form is subject to change (see \textcite{eutxo} for the details), but it should contain all the information about the inputs, outputs, metadata, etc.

\subsubsection{Ensuring determinism}
The information provided by the \gls{context} is entirely determined by the transaction itself.
This ensures that we satisfy \cref{req:ledger-determinism}.

We handle the issue of time by providing only the validation interval of the transaction.
This serves to give the \gls{validator} a window within which the current time must lie: in practice this is enough to establish that the current time is ``definitely before'' or ``definitely after'' some particular time, which is the most common kind of constraint.

Some additional responsibility passes to the author of the transaction as a result of this.
If the \gls{validator} they are trying to satisfy requires that the transaction occur before some time $T$, then they must not only submit the transaction before $T$, they must ensure that the upper bound of the validity interval is below $T$.
This doesn't make the transaction less likely to validate (it would not have validated after $T$ regardless!), but it does require some more work from the \gls{off-chain}.

\subsubsection{EUTXO-1 validity}
\label{sec:eutxo-1-valid}

In addition to the conditions in \cref{sec:utxo-valid}, the following condition must hold:
\begin{itemize}
\item Scripts must succeed: if a transaction spends a \gls{script-output}, then the \gls{validator} on the output must be run with the \gls{redeemer}, \gls{datum}, and \gls{context} and return successfully.
\end{itemize}

\subsection{EUTXO-2: \glsentrytext{multicurrency}}
\label{sec:multicurrency}

\todompj{This section is also a mess and doesn't tell the story well.}

The other extension that we make to the ledger is to enable \emph{\gls{multicurrency}}, that is, to enable our ledger to represent a wider variety of assets than a single one.
\Gls{multicurrency} is a key part of the \gls{plutus-platform} not only because custom \glspl{currency} are an important use case for blockchain systems, but because we use lightweight custom \glspl{token} as integral parts of our application designs.

The key design innovation is that \glspl{currency} are identified with the hash of the \gls{mps} that controls them.
This means that to be allowed to forge \glspl{token} of a particular \gls{currency}, you must present the corresponding \gls{mps} with the transaction and it must pass.

This design allows our system to be both \emph{local} and \emph{lightweight}:
\begin{itemize}
\item
  We do not require any kind of central registration of \glspl{currency}: any hash can be used as a \gls{currency-id}, you just cannot forge any unless you can produce a \gls{script} with that hash.
  This ensures we do not violate \cref{req:ledger-locality}.
\item
  \Gls{forging} \glspl{token} is cheap: all that is required is to provide and run the script at the time of forging.
\item
  Transacting with custom \glspl{currency} is cheap: they are natively supported and work just like the native token.
\end{itemize}

\todompj{Say more, maybe write out some of the requirements separately.}

\subsubsection{The \glsentryname{value} type}
\label{sec:value}
The first change we make is to our notion of ``value'' on the ledger.
In traditional \gls{utxo} ledgers, value is just an integer representing a quantity of the (unique) asset tracked by the ledger.

We generalize this to support multiple asset classes as follows: we define a type \gls{value} which represents a \emph{bundle} of assets, indexed by a \emph{\gls{currency-id}} and then by a \emph{\gls{token-name}}.

\begin{quote}
\begin{verbatim}
type Value = Map CurrencyID (Map TokenName Quantity)
\end{verbatim}
\end{quote}

The \gls{currency-id} indicates the \emph{controller} of the currency, specifically it is the hash of the \gls{mps} that controls when tokens of that currency may be forged or destroyed (see \cref{sec:forging}).
The \gls{token-name} separates different \emph{classes} of \gls{token} within the currency. Only \glspl{token} of the same currency and \gls{token-name} are \gls{fungible} with each other.
Hence one could use a currency to track items in a game: different kinds of items would have different \glspl{token-name}, and an asset bundle could consist of e.g. 3 swords, 7 hats, and a cheesecake.

Operations on \gls{value} are defined as if it were a finitely-supported function.

The ledger rules do not need to change much in order to support this generalized \gls{value}: the only change is that we need to use the appropriate sum operation on \gls{value} in e.g. the balancing rules.

\subsubsection{\Glsentrytext{forging}}
\label{sec:forging}
\Gls{value} may also be created and destroyed, we call this \emph{\gls{forging}}.
To this end transactions are given two additional fields:
\begin{enumerate}
\item \code{forge} which contains \gls{value}
\item \code{forgeScripts} which contains \glspl{mps} for all the \glspl{currency} forged in \code{forge}
\end{enumerate}

\subsubsection{\glsentrylong{mps}s}
The \glspl{script} which control the forging of \glspl{token} are called \glsfirstplural{mps}.
Since these are able to see the \gls{context}, and hence the whole transaction being validated, they can enforce a wide range of properties on the transaction.

\subsubsection{\glsentrylong{nft}s}
A \gls{nft} is a unique \gls{token} which can be transferred to another user, but not duplicated.
\Glspl{nft} have proven useful in a number of blockchain applications (see \textcite{ERC-721}); for example, they can represent ownership of some object in a game, or shares in a company, or many other kinds of asset.
We can implement \glspl{nft} as token classes within a \gls{currency} whose supply is limited to a single \gls{token}.

Ensuring that the tokens are unique is not trivial, since uniqueness is a global property and transactions only have access to local information.
Two approaches are:
\begin{itemize}
\item
  Use a \gls{one-shot} \gls{mps}.
  This ensures that the \gls{mps} can only be run once, and then so long as you do not issue any duplicates there, you will have uniqueness.
\item
  Run a unique state machine (beginning with a \gls{one-shot} transaction to ensure uniqueness) which keeps all the issued tokens in its state.
  This effectively introduces global state that must be consulted before each forging, allowing uniqueness to be maintained.
\end{itemize}

\subsubsection{EUTXO-2 validity}
\label{sec:eutxo-2-valid}

In addition to the conditions in \cref{sec:utxo-valid} and \cref{sec:eutxo-1-valid}, the following condition must hold:
\begin{itemize}
\item Forging must be authorized: for each \gls{currency-id} of which a non-zero amount is forged in \code{forge}, the corresponding \gls{mps} must be run with the \gls{context} and return successfully.
\end{itemize}
