%% Plutus Core Specification

\documentclass[a4paper]{article}
% Let's try full width text to see if it makes 
% it easier to get figures in the right place

\usepackage{blindtext, graphicx}
\usepackage{url}

% *** MATH PACKAGES ***
%
\usepackage[cmex10]{amsmath}
\usepackage{stmaryrd}

% *** ALIGNMENT PACKAGES ***
%
\usepackage{array}
\usepackage{float}  %% Try to improve placement of figures.  Doesn't work well with subcaption package.
\usepackage{subcaption}
\usepackage{caption}

% correct bad hyphenation here
\hyphenation{}

\usepackage{subfiles}
\usepackage{geometry}
\usepackage{pdflscape}

% *** IMPORTS FOR PLUTUS LANGUAGE ***

\usepackage[T1]{fontenc}
\usepackage{bussproofs,amsmath,amssymb}
\usepackage{listings}
\usepackage{xcolor}


% *** DEFINITIONS FOR PLUTUS LANGUAGE ***

%%% General Misc. Definitions

\newcommand{\red}[1]{\textcolor{red}{#1}}
\newcommand{\redfootnote}[1]{\red{\footnote{\red{#1}}}}
\newcommand{\blue}[1]{\textcolor{blue}{#1}}
\newcommand{\bluefootnote}[1]{\blue{\footnote{\blue{#1}}}}


\newcommand{\diffbox}[1]{\text{\colorbox{lightgray}{\(#1\)}}}
\newcommand{\judgmentdef}[2]{\fbox{#1}

\vspace{0.5em}

#2}
\newcommand{\hyphen}{\operatorname{-}}
\newcommand{\repetition}[1]{\overline{#1}}
\newcommand{\Fomega}{F$^{\omega}$}
\newcommand{\keyword}[1]{\texttt{#1}}
\newcommand{\construct}[1]{\texttt{(} #1 \texttt{)}}



%%% Term Grammar

\newcommand{\sig}[3]{[#1](#2)#3}
\newcommand{\constsig}[2]{#1,#2}
\newcommand{\con}[1]{\construct{\keyword{con} ~ #1}}
\newcommand{\abs}[3]{\construct{\keyword{abs} ~ #1 ~ #2 ~ #3}}
\newcommand{\inst}[2]{\texttt{\{}#1 ~ #2\texttt{\}}}
\newcommand{\lam}[3]{\construct{\keyword{lam} ~ #1 ~ #2 ~ #3}}
\newcommand{\app}[2]{\texttt{[} #1 ~ #2 \texttt{]}}
\newcommand{\wrap}[3]{\construct{\keyword{wrap} ~ #1 ~ #2 ~ #3}}
\newcommand{\unwrap}[1]{\construct{\keyword{unwrap} ~ #1}}
\newcommand{\builtin}[3]{\construct{\keyword{builtin} ~ #1 ~ #2 ~ #3}}
\newcommand{\error}[1]{\construct{\keyword{error} ~ #1}}
\newcommand{\sizedBuiltin}[2]{#1\keyword{!}#2}



%%%  Type Grammar

\newcommand{\funT}[2]{\construct{\keyword{fun} ~ #1 ~ #2}}
\newcommand{\fixT}[2]{\construct{\keyword{fix} ~ #1 ~ #2}}
\newcommand{\allT}[3]{\construct{\keyword{all} ~ #1 ~ #2 ~ #3}}
\newcommand{\conIntegerType}[1]{\keyword{integer}}
\newcommand{\conBytestringType}[1]{\keyword{bytestring}}
\newcommand{\conSizeType}[1]{\keyword{size}}
\newcommand{\builtinT}[2]{\construct{\keyword{builtin} ~ #1 ~ #2}}
\newcommand{\conT}[2]{\construct{\keyword{con} ~ #1 ~ #2}}
\newcommand{\lamT}[3]{\construct{\keyword{lam} ~ #1 ~ #2 ~ #3}}
\newcommand{\appT}[2]{\texttt{[} #1 ~ #2 \texttt{]}}

\newcommand{\typeK}{\construct{\keyword{type}}}
\newcommand{\funK}[2]{\construct{\keyword{fun} ~ #1 ~ #2}}
\newcommand{\sizeK}{\construct{\keyword{size}}}



%%% Program Grammar

\newcommand{\version}[2]{\construct{\keyword{version} ~ #1 ~ #2}}



%%% Judgments

\newcommand{\hypJ}[2]{#1 \vdash #2}
\newcommand{\ctxni}[2]{#1 \ni #2}
\newcommand{\validJ}[1]{#1 \ \operatorname{valid}}
\newcommand{\termJ}[2]{#1 : #2}
\newcommand{\typeJ}[2]{#1 :: #2}
\newcommand{\istermJ}[2]{#1 : #2}
\newcommand{\istypeJ}[2]{#1 :: #2}



%%% Contextual Normalization

\newcommand{\ctxsubst}[2]{#1\{#2\}}
\newcommand{\typeStep}[2]{#1 ~ \rightarrow_{ty} ~ #2}
\newcommand{\typeMultistep}[2]{#1 ~ \rightarrow_{ty}^{*} ~ #2}
\newcommand{\typeBoundedMultistep}[3]{#2 ~ \rightarrow_{ty}^{#1} ~ #3}
\newcommand{\step}[2]{#1 ~ \rightarrow ~ #2}
\newcommand{\multistepIndexed}[3]{#1 ~ \rightarrow^{#2} ~ #3}
\newcommand{\normalform}[1]{\lfloor #1 \rfloor}
\newcommand{\subst}[3]{[#1/#2]#3}
\newcommand{\kindEqual}[2]{#1 =_{\mathit{k}} #2}
\newcommand{\typeEqual}[2]{#1 =_{\mathit{ty}} #2}
\newcommand{\typeEquiv}[2]{#1 \equiv_{\mathit{ty}} #2}


\newcommand{\inConT}[2]{\conT{#1}{#2}}
\newcommand{\inAppTLeft}[2]{\appT{#1}{#2}}
\newcommand{\inAppTRight}[2]{\appT{#1}{#2}}
\newcommand{\inFunTLeft}[2]{\funT{#1}{#2}}
\newcommand{\inFunTRight}[2]{\funT{#1}{#2}}
\newcommand{\inAllT}[3]{\allT{#1}{#2}{#3}}
\newcommand{\inFixT}[2]{\fixT{#1}{#2}}
\newcommand{\inLamT}[3]{\lamT{#1}{#2}{#3}}

\newcommand{\inBuiltin}[5]{\builtin{#1}{#2}{#3 #4 #5}}



%%% CK Machine Normalization

\newcommand{\cksteps}[2]{#1 ~\mapsto~ #2}
\newcommand{\ckforward}[2]{#1 \triangleright #2}
\newcommand{\ckbackward}[2]{#1 \triangleleft #2}
\newcommand{\ckerror}{\blacklozenge}

\newcommand{\inInstLeftFrame}[1]{\inst{\_}{#1}}
\newcommand{\inWrapRightFrame}[2]{\wrap{#1}{#2}{\_}}
\newcommand{\inUnwrapFrame}{\unwrap{\_}}
\newcommand{\inAppLeftFrame}[1]{\app{\_}{#1}}
\newcommand{\inAppRightFrame}[1]{\app{#1}{\_}}



\begin{document}
%
% paper title
% can use linebreaks \\ within to get better formatting as desired
\title{Formal Specification of\\the Plutus Core Language v1.0 (RC5.5)}


\maketitle

\thispagestyle{plain}
\pagestyle{plain}


%\begin{abstract}
%\boldmath
%The Plutus Language is outlined, together with the major
%design decisions for implementations. A formal specification of the
%language is given, including an elaborator and bidirectional type
%system.

\section{Plutus Core}

Plutus Core is a typed, eagerly evaluated $\lambda$-calculus designed
for use as a transaction validation language in blockchain
systems. More precisely, Plutus Core is the system $F_\omega$ of
Girard and Reynolds (see \cite{Girard-thesis},
\cite{Reynolds-type-structure}, \cite[\S30]{Pierce:TAPL}) with a number
of extensions (a type-level fixpoint operator, isorecursive types, and 
a library of basic types and functions).  Plutus Core is
meant to be a compilation target, and this is reflected in the design
of the language: while writing large Plutus Core programs by hand is
difficult, the language is relatively straightforward to formalize
with a proof assistant.

We have tried keep the Plutus Core language small. There is no
explicit support for algebraic datatypes, but they are representable
using, for example, the Scott encoding (\cite{Scott-encoding}: see
\cite{Koopman:2014} for example; note that the Scott enconding
coincides with the Church encoding for non-recursive
types). Similarly, there are no explicit constructs for recursion or
branching, but the type system is sufficiently expressive to allow us
to use standard combinators (for example the $Z$ combinator and
Church/Scott encoded booleans) for such purposes.


\subsection{Blockchain issues}
Plutus Core code (and code intended for execution on a blockchain in
general) can be executed in two different environments:
\textit{off-chain} and \textit{on-chain}.  As the name suggests,
off-chain execution doesn't happen on the blockchain itself, but in
some other environment, such as in an electronic wallet on a
smartphone or PC.  In contrast, on-chain execution takes place on
\textit{core nodes}, machines which are actually maintaining the
blockchain.  It is important that core nodes process transactions
quickly and efficiently; moreover, core nodes benefit from executing
transactions and compete with each other to do so.  Denial of service
thus becomes an issue for on-chain code, and it is important that a
malicious user cannot submit code which consumes excessive amounts of
processor time or memory.

To deal with this problem, a charge is levied when a core node runs
on-chain code.  Units of so-called \textit{gas} are consumed as a
program runs, and users must buy gas in advance.  

Because of these issues, it is important that the cost of on-chain
code execution is easy to calculate, certainly during execution (so
that gas consumption can be accounted for), but ideally also in
advance so that a user has some idea of how much a contract is likely
to cost and can make sure that they buy enough gas to allow their
contract to run to completion.  The design of Plutus Core incorporates
a number of features to facilitate cost analysis for this purpose,
notably explicitly \textit{sized} types for integers and bytestrings.

Resource limitations also preclude complicated on-chain code analysis
and verification.  We perform type checking prior to execution of
Plutus Core Code, but no other validation.  Type checking itself takes
time and requires memory, so there is also a gas charge for on-chain
type checking.

\section{Syntax}

The grammar of Plutus Core is given in
Figures~\ref{fig:Plutus_core_lexical_grammar} and
\ref{fig:Plutus_core_grammar}. This grammar describes the abstract
syntax trees of Plutus Core in a convenient notation, and also
describes the string syntax to be used when referring to those
ASTs. The string syntax is not fundamental to Plutus Core, and only
exists because we must refer to programs in documents such as
the present one. Plutus Core programs are intended to exist only as ASTs produced by
compilers from higher languages, and as serialized representations on
blockchains, and therefore we do not expect anyone to write programs
in Plutus Core, nor need to use a parser for the language.

Lexemes are described in standard regular expression notation.  The only other
lexemes are round \texttt{()}, square \texttt{[]}, and curly \texttt{\{\}}
brackets.  Spaces and tabs are allowed anywhere, and have no effect
save to separate lexemes.

Application in both terms and types is indicated by square
brackets, and instantiation in terms is indicated by curly brackets. We
permit the use of multi-argument application and instantiation as
syntactic sugar for iterated application.
For instance,
\[
  [M_0 ~ M_1 ~ M_2 ~ M_3]
\]
is sugar for
\[
  [[[M_0 ~ M_1] ~ M_2] ~ M_3]
\]
All subsequent definitions assume iterated application and instantiation
has been expanded out, and use only the binary form. To the extent that
a standard utility parser for Plutus Core might be made, for debugging
purposes and other such things, iterated application and instantiation
ought to be included as sugar.

%In this grammar, we have multi-argument application, both in types (\(\appT{A}{B^+}\)) and in terms (\(\app{M}{N^+}\)), as well as multi-argument instantiation in terms (\(\inst{M}{A^*}\)). This is to be understood as a convenient form of syntactic sugar for iterated binary application associated to the left, and the formal rules treat only the binary case.

\subfile{figures/PlutusCoreLexicalGrammar}

\subfile{figures/PlutusCoreGrammar}

\newcommand\fixtype[1]{\mu\,\alpha.#1}  % Just for talking about the fix operator in the notes.

\subsection{Notes on the grammar}
\paragraph{Terms.}
\begin{itemize}
\item $\con{cn}$ represents a built-in constant: see \S\ref{sec:builtins}
\item $\abs{\alpha}{K}{V}$ represents a polymorphic value abstracted
  over a type; this would often be donated by $\Lambda\alpha{::}K.V$.
  Note that we require the body of the abstraction to be a
  \textit{value}, not a term: see the note on the value restriction below.
\item $\inst{M}{A}$ represents a polymorphic term instantiated at a particular type.
\item $\wrap{\alpha}{A}{V}$ and $\unwrap{M}$: see the note on recursive types below.
\item $\lam{x}{A}{M}$ is standard lambda-abstraction: $\lambda{}x{:}{A}.{M}$.
\item $\app{M}{N}$ is standard function application.
\item $\builtin{bn}{A^*}{M^*}$ is application of a built-in function: see \S\ref{sec:builtins} for more information.
\item $\error{A}$ causes an error, terminating computation immediately.
\end{itemize}

%% Re the value restriction:  Phil says
% Value restriction: This guarantees that one can erase a type
% abstraction to its body (otherwise, one needs to erase a type
% abstraction to a value abstraction). It also corresponds to the value
% restriction in Standard ML, which ensures polymorphism works with
% effects. For instance, if we wanted to translate Plutus into a monad
% to make non-termination and reading the environment explicit, that is
% easy with the value restriction; otherwise, it is (so far as I know)
% not known how to do it. The restriction appears in my paper Theorems
% for Free for Free with Ahmed, Jamner, and Siek in ICFP 2017.



\paragraph{Values.} Values are terms which cannot undergo any further reduction.

\paragraph{Types.} Plutus Core contains a copy of the simply typed lambda calculus
  at the type level, together with a few extensions.
\begin{itemize}
\item Sizes $s$ are natural numbers used to enforce bounds on certain types of values: see below.
\item $\funT{A}{B}$ is the type of functions from $A$ to $B$, $A \rightarrow B$.
\item $\allT{\alpha}{K}{A}$ represents the type of a polymorphic term (eg a type abstraction), $\forall \alpha{::}K.A$.
\item $\fixT{\alpha}{A}$ is a type-level fixpoint operator $\fixtype{A}$: see the note on recursive types below.
\item $\lamT{\alpha}{K}{A}$ is abstraction of types over types, $\lambda \alpha{::}K.A$.
\item $\appT{A}{B}$ is function application at the type level.
\item $\conT{tcn}{A}$ represents a built-in type, and $A$ will often be a size.  For example, $\conT{\textrm{integer}}{8}$
is the type of 8-byte integers.  See \S\ref{sec:builtins} for more information.
\end{itemize}

\paragraph{Type Values.} All types can be reduced to a type value (or
  \textit{normalised type}) which cannot undergo any further
  reduction: see section~\ref{sec:reduction}.  The type normalisation
  process always terminates: the proof is similar to the proof of the
  fact that reduction of terms in the simply typed lambda calculus
  always terminates (\cite[\S12]{Pierce:TAPL}).  The type-level
  structure of Plutus Core is slightly more complicated than the
  simply typed lambda calculus, but the extra forms (\texttt{fix},
  \texttt{all}, built-in base types) don't provide any new
  opportunities for type reduction and thus have little effect on the
  proof.

\paragraph{Kinds.} In Plutus Core we have a copy of the 
simply typed lambda calculus at the type level. The types of the
simply typed lambda calculus are lifted to the level of
\textit{kinds} in Plutus Core, allowing the type system to talk about
operations which themselves occur at the level of types. In addition
to the standard kind \texttt{(type)} (usually written as $\star$), we
have a kind \texttt{(size)} which allows us to deal with the sized
types described below in~\S\ref{sec:size-note}.


\paragraph{Programs and Versions.} A complete Plutus Core program 
consists of a standard version number (1.0.2, for example) indicating
the Plutus Core version and a \textit{closed} term (i.e., a term with no
free variables) forming the body of the program.  The version number
is used by the Plutus Core evaluator and other tools to check that
they are dealing with code conforming to the correct version of the
Plutus Core language.

\subsection{More detailed remarks}
\subsubsection{Type Normalisation}
 Type checking requires normalisation to be performed, for example to
 check that two types are equal.  This process can be expensive, so we
 require that code supplied for on-chain execution comes with all type
 annotations pre-normalised.  This can be done during off-chain type
 checking.  Another issue here is that normalisation can cause sizes
 to increase (in the worst case, exponentially \blue{(really?)}), and
 requiring normalisation to be performed before a program is submitted
 to the chain facilitates prediction of on-chain costs.  \blue{Is
   there an issue that further normalisation may be required when
   programs are composed on-chain?}

\subsubsection{Sizes and built-in types and functions}
\label{sec:size-note}
Details of built-in types and functions are given
in~\S\ref{sec:builtins}, but a remark on sizes may be helpful at this
point.  Plutus Core currently provides two types of built-in data with
variable sizes: integers and bytestrings (blocks of binary data used
for cryptographic purposes).  There is no fixed upper bound on the
size of values of these types, but -- as mentioned in the introduction -- it
could be problematic if users were allowed to create values of arbitrary size 
at runtime. To avoid this, sizes are tracked in the type system so that 
appropriate charges can be made. 

For integers, this works as follows (the system for 
bytestrings is similar).  Integers are signed values:
there is no fixed upper bound, but values are represented as sequences
of a specified number of 8-bit bytes.  For example, \texttt{1~!~100}
is the integer 100 stored as a single byte.  Literal constants must
fit into the specified size: \texttt{1~!~200} gives a compile-time
error. If a value overflows at runtime then an error occurs and
execution is terminated.  Integers are manipulated using built-in
functions (see Figure~\ref{fig:Plutus_core_builtins}) whose types
include the size. This prevents users from writing programs which
generate arbitrarily large values during execution; it also allows us
to charge according to the actual amount of computation performed,
which will increase as the size of the integers involved increases.


\subsubsection{Recursive types}
The type-level fixpoint operator $\fixT{\alpha}{A}$ allows one to
define recursive types in Plutus Core: $\fixT{\alpha}{A}$ is a type
$\fixtype{A}$ such that $\fixtype{A} \cong [\fixtype{A}/\alpha]A$; in
practice the type $A$ here will usually be a function type involving
$\alpha$.  Note that we have an \textit{isomporhism} of types here
rather than equality.  Fixpoint types are
\textit{isorecursive}~\cite[20.2]{Pierce:TAPL} with explicit maps
\texttt{wrap}: $[\fixtype{A}/\alpha]A \rightarrow \fixtype{A}$ and
\texttt{unwrap}: $\fixtype{A} \rightarrow [\fixtype{A}/\alpha]A$.
The use of isorecursive types makes it somewhat more difficult to
write \textit{terms}, but makes it much easier to reason about
\textit{types}, and in particular simplifies the typechecking process
considerably.

\noindent\blue{Do we want an example here?  Also, ``isorecursive'' or ``iso-recursive''? 
Is it true that we have an isomporhism, and not a retraction or something?}

% As an example, consider the program in Figure
% \ref{fig:Plutus_core_example}, which defines the type of natural
% numbers as well as lists, and the factorial and map functions. This
% program is not the most readable, which is to be expected from a
% representation intended for machine interpretation rather than human
% interpretation, but it does make explicit precisely what the roles
% are of the various parts.

%\subfile{figures/PlutusCoreExample}


\subsubsection{The value restriction}
In $\abs{\alpha}{K}{V}$ we require the body $V$ to be a value. We will
refer to this restriction as the \textit{value restriction} because it
can be regarded as a generalisation of the value restriction in
Standard ML (\cite[22.7]{Pierce:TAPL}). Experience has shown that
allowing general terms as bodies of type abstractions can cause a
number of difficulties.  For example, Standard ML's value restriction
is required to avoid a number of problems when (implicit) type
abstractions include non-values; other issues are explored
in~\cite[2.4]{Ahmed:2017}. Plutus Core's restriction to values avoids
these problems and is not very onerous in practice.
  


\section{Type Correctness}

We define for Plutus Core a number of typing judgments which explain ways that a program can be well-formed. First, in Figure \ref{fig:Plutus_core_contexts}, we define the grammar of variable contexts that these judgments hold under. Variable contexts contain information about the nature of variables --- type variables with their kind, and term variables with their type.

Then, in Figure \ref{fig:Plutus_core_kind_synthesis}, we define what it means for a type to synthesize a kind. Plutus Core is a higher-kinded version of System F, so we have a number of standard System F rules together with some obvious extensions. In Figure \ref{fig:Plutus_core_type_synthesis}, we define the type synthesis judgment, which explains how a term synthesizes a type.

Finally, type synthesis for constants ($\con{bn}$ and $\conT{bt}\!\!$)
% \!: negative spaces to get rid of superfluous space
is given in tabular form rather than in inference rule form, in Figure \ref{fig:Plutus_core_builtins}, which also gives the reduction semantics. This table also specifies what conditions trigger an error.





\subfile{figures/PlutusCoreTypeCorrectness}


\section{Reduction and Execution}
\label{sec:reduction}

In this section we define a standard eager,
small-step contextual semantics~\cite[5.3]{Harper:PFPL} (or
\textit{reduction semantics}~\cite[\S2]{Felleisen-Hieb}) for Plutus
Core in terms of the reduction relation for types
(\(\typeStep{A}{A'}\)) (Figure~\ref{fig:type-reduction}) 
and terms (\(\step{M}{M'}\)) (Figure~\ref{fig:term-reduction}), which
incorporates both $\beta$ reduction and contextual congruence. We make
use of the transitive closure of these stepping relations via the
usual Kleene star notation.
% It seems that Harper talks about "contextual semactics" in the 
% first edition of his book, but changed it to "contextual dymanics"
% in the second.  It doesn't seem to be a standard term elsewhere though.

In the context of a blockchain system, it can be useful to also have a
step indexed version of stepping, indicated by a superscript count of
steps (\(\multistepIndexed{M}{n}{M'}\)). In order to prevent
transaction validation from looping indefinitely, or from simply
taking an inordinate amount of time, which would be a serious security
flaw in the blockchain system, we can use step indexing to put an
upper bound on the number of computational steps that a program can
have. In this setting, we would pick some upper bound $\mathit{max}$
and then perform steps of terms $M$ by computing which $M'$ is such
that \(\multistepIndexed{M}{\mathit{max}}{M'}\).


\newpage 

\subsection{Type reduction}
Because the Plutus Core type system contains a copy of the simply
typed lambda calculus, complex computations can take place at the
level of types.  Reductions in the type system always transform a
given type into an equivalent type, and so have no effect on the term
level.  In conjunction with the fact that type reduction always
terminates, this allows us to perform normalisation (i.e., reduction to
a form which cannot undergo any further reduction) statically, before
execution begins.  Figure~\ref{fig:type-reduction} contains the rules
for type reduction.

\subfile{figures/PlutusCoreTypeReduction}


\subsection{Term reduction}
Execution of Plutus Core programs is performed by (possibly
non-terminating) reduction of well-typed terms.  The reduction rules
are contained in Figure~\ref{fig:term-reduction}, and give us a fairly
standard operational semantics.

\subfile{figures/PlutusCoreTermReduction}

\subsection{An abstract machine for evaluating Plutus Core programs}
This section contains a description of an abstract machine for
executing Plutus Core.  This is based on the CK machine of Felleisen
and Friedman~\cite{Felleisen-CK-CEK}. 

This machine is intended as a reference implementation which is
amenable to formalisation, ideally so that it can be proved to
implement the operational semantics described above.  The CK machine
is inefficient because it implements application as $\beta$-reduction,
so that when evaluating $(\lambda x.M)V$, $V$ must be substituted
bodily for $x$ wherever it occurs in $M$.  This process can require
considerable amounts of time and space.  More efficient machines are
available (for example the CEK machine from~\cite{Felleisen-CK-CEK},
which uses environments binding actual arguments to variables), and in
practice we would use something based on such a machine; however, the
CK machine will still provide a useful bridge for formalisation
purposes.

\subfile{figures/PlutusCoreCkMachine}

\noindent The machine alternates between two main phases: \textit{computation}
($\triangleright$), where it recurses down the AST looking for values,
saving surrounding contexts as frames (or \textit{reduction contexts})
on a stack as it goes; and \textit{return} ($\triangleleft$), where it
has obtained a value and pops a frame off the stack to tell it how to
proceed next.  In addition there is an error state $\blacklozenge$
which halts execution with an error, and a stop state $\square$ which
halts execution and returns a value to the outside world.

\blue{When a builtin returns a result, should it be a value
  that gets $\triangleleft$ or a term that gets $\triangleright$?  The 
  new dynamic built-in stuff would probably require the latter.} 

\section{Built in types, functions, and values}
\label{sec:builtins}
Plutus Core comes with a pre-defined set of built-in types and
functions which will be useful for blockchain applications.  The set
of built-ins is fixed, although in future we may provide a framework
to allow customisation for specialised blockchains.

As mentioned earlier, there are three basic types: \texttt{size},
\texttt{integer}, and \texttt{bytestring}.  The \textrm{size} type is
used to track the sizes of objects, the \texttt{integer} type, in
conjunction with \texttt{size} provides integers of specified sizes,
and the \texttt{bytestring} type provides sequences of bytes, again of
specified sizes.  These types are given in
Figure~\ref{fig:Plutus_core_type_constants}:  for example 
\texttt{(builtin integer 10)} is the type of signed 10-byte integers.

\red{Is it \texttt{con} or \texttt{builtin}?  I think there's a 
conflict between Figure~\ref{fig:Plutus_core_type_abbreviations}
and Figure~\ref{fig:Plutus_core_grammar}.}

Terms of these types are written using the \texttt{!} character, as
shown in Figure~\ref{fig:Plutus_core_constants}.  For example
\texttt{10~!~12345} is a ten-byte integer whose value is 12345 (and
\textrm{1~!~12345} would cause an error), and \texttt{20~!~\#3ac29d873246f} 
is a 20-byte bytestring containing the bytes
shown\footnote{In fact, the bounds are upper bounds, not actual sizes:
  in the current implementation, \texttt{10~!~12345} will only take up
  as much space as it requires, not the full 10 bytes.}.  We use the
same syntax to denote expressions of appropriate types in
Figure~\ref{fig:Plutus_core_builtins}.

We provide standard arithmetic and comparison operations for integers
and a number of list-like functions for bytestrings. The details are
given in Figure~\ref{fig:Plutus_core_builtins}, using a number of
abbreviations given in Figure~\ref{fig:Plutus_core_type_abbreviations}.
Note the following:
\begin{itemize}
\item The signatures of the functions statically limit the way in
  which sizes are used.  It is not possible to create values of
  unlimited size at runtime.  If any function attempts to create a
  value exceeding the expected size, an error will occur during execution.
\item We use fixed Scott encodings for certain types and values,
  specifically for the \texttt{unit} and \texttt{boolean} types.
Compilers targeting Plutus Core must be aware of these encodings and
use them appropriately.
\end{itemize}

\newpage

\subfile{figures/PlutusCoreBuiltins}

\section{Examples}

\subfile{figures/PlutusCoreExampleAgain}

\section{Basic Validation Program Structure}

The basic way that validation is done in Plutus Core is somewhat
similar to what's in Bitcoin Script. Whereas in Bitcoin Script, a
validation is successful if the validating script successfully
executes and leaves $\textit{true}$ on the top of the stack, in Plutus
Core, validation is successful when the script reduces to any value
other than \(\error{A}\) in the allotted number of steps.


%\section{Erasure}

%TO WRITE

%\subfile{figures/PlutusCoreErasureGrammar}

%\subfile{figures/PlutusCoreErasureReduction}

%\subfile{figures/PlutusCoreErasureTheorem}

%\section{Example}

%\subfile{figures/PlutusCoreExampleAgain}



\section{Unrestricted Algorithmic Type System}

For implementation purposes, it's useful to have a variant of the type system that is more algorithmic in its presentation. We give this here, showing the figures that have been changed (relative to the declarative version above), and highlighting the specific parts of each rule that is different, where possible.

\subfile{figures/PlutusCoreTypeCorrectness-Algorithmic-Unrestricted.tex}



\section{Restricted Algorithmic Type System}

For blockchain purposes, it is useful to not only have an algorithmic system, but to require that no un-normalized types are present in programs, so as to reduce the amount of computation necessary to typecheck a program. It's also useful to have a way of bounding the amount of computation that can be done in type checking, for security purposes. To that end, we present a restricted grammar and type reduction, and a type system that reflects these. The figures below are additions with respect to the unrestricted version, and show only the new or changed figures and their respective highlighted differences.

\subfile{figures/PlutusCoreGrammar-Algorithmic-Restricted.tex}

\subfile{figures/PlutusCoreReduction-Algorithmic-Restricted.tex}

\subfile{figures/PlutusCoreTypeCorrectness-Algorithmic-Restricted.tex}

\bibliographystyle{plain} %% ... or whatever
\bibliography{plutus-core-specification}

\end{document}
