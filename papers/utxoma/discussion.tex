\section{Related work}

\paragraph{Ethereum.}

Ethereum's ERC token standards~\cite{erc20,erc721} are one of the better known multi-asset
implementations. They are a non-native implementation and so come with a set of drawbacks,
such as having to implement ledger functionality (such as asset transfers) using smart contracts, rather
than using the underlying ledger.

Augmenting the Ethereum ledger with functionality similar to that of the model
we present here would likely be possible. Additionally, access to global
contract state would make it easier to define forging policies that care about
global information, such as the total supply of an asset.

\paragraph{Waves.}

Waves~\cite{waves} is an account-based multi-asset ledger, supporting its own smart contract
language. In Waves, both accounts and assets themselves can
be associated with contracts. In both cases, the association is made by
adding the associated script
to the account state (or the state of the account containing the asset).
A script associated with an asset imposes conditions on the use of this asset,
including minting and burning restrictions, as well as transfer restrictions.

\paragraph{Stellar.}

Stellar~\cite{stellar} is an account-based native multi-asset system geared towards tracking
real-world item ownership via associated blockchain tokens.
Stellar is optimised to allow a token issuer to maintain a level of control over
the use of their token even once it changes hands. The Stellar ledger also features
a distributed exchange listing, which is used to
facilitate matching (by price) exchanges between different tokens.

\paragraph{Zilliqa.}

Zilliqa
is an account-based platform with an approach to smart contract implementation
similar to that of Ethereum~\cite{scilla-arxiv}.
The Zilliqa fungible and non-fungible tokens are
designed in a way that mimics the ERC-20 and ERC-721 tokens, respectively.
While this system is designed to be statically analysable, it does not offer new solutions to
the problem of dependency on the global state.

\paragraph{DAML.}

DAML~\cite{daml} is a smart contract language designed to be used on the DAML ledger model.
The DAML ledger model does not support keeping records of asset ownership, but instead,
only stores current contract states in the following way:
a valid transaction interacting with a contract results in the creation of new
contracts that are the next
steps of the original contract, and removal of the original contract from the ledger.

Only the contracts with which a transaction
interacts are relevant to validating it, which is similar to our approach
of validation without global context.
Although this system does not have built-in multi-asset support (or any
ledger-level accounting),
the transfer of any type of asset can be represented on the ledger via contracts.
Due to the design of the system to operate entirely by listing contracts on the ledger,
each
action, including accepting funds transferred by a contract, requires consent. This
is another significant way
in which this model is different from ours.

%\todompj{This doesn't actually say anything about multi-asset stuff?}

\paragraph{Bitcoin.}

Bitcoin popularised \UTXO\ ledgers, but has neither native nor non-native
multi-asset support on the main chain.
The Bitcoin ledger model does not appear to have the accounting infrastructure
or sufficiently expressive
smart contracts for implementing multi-asset support in a generic way.
There have been several layer-two approaches to implementing custom Bitcoin assets.
Because each mined Bitcoin is unique, a particular
Bitcoin can represent a specific custom asset, as is done in~\cite{coloredcoins}.
A more sophisticated accounting strategy is implemented in another layer-two
custom asset approach~\cite{nick2020liquid}.
There have also been attempts to implement custom tokens using Lightning network
channels~\cite{lntokens}.

\paragraph{Tezos.}

Tezos~\cite{tezos} is an account-based platform with its own smart contract language.
It has been used to implement an ERC-20-like fungible token standard (FA1.2),
with a unified token standard in the works (see~\cite{tezosMA}).
The custom tokens for both multi-asset standards are non-native, and thus have shortcomings
similar to those of Ethereum token standards.

%\todompj{This is very vague.}

\paragraph{Nervos CKB.}

Nervos CKB~\cite{nervos} is a \UTXO{}-inspired platform that operates on a broader notion of a Cell,
rather than the usual output balance amount and address, as the value stored in
an entry. A Cell entry can contain any type of data,
including a native currency balance, or any type of code.
This platform comes with a Turing-complete scripting language that can be used to define custom
native tokens. There is, however, no dedicated accounting infrastructure to handle
trading custom assets in a similar way as the base currency type.

\section{Discussion}
\label{sec:discussion}

\subsection{General observations}
\subsubsection{Asset registries and distributed exchanges.}

The most obvious way to manage custom assets might be to add some kind of global \emph{asset registry}, which associates a new asset group with its policy.
Once we have an asset registry, this becomes a natural place to put other kinds of infrastructure that rely on global state associated with assets, such as decentralised exchanges.

However, our system provides us with a way to associate forging policies and the assets controlled by them \emph{without} any global state. This simplifies the ledger implementation in the concurrent and distributed environment of a blockchain.
Introducing global state into our model would result in disrupted synchronisation (on which \cite{chakravarty2020hydra} relies to great effect to implement fast, optimistic settlement), as well as slow and costly state updates at the time of asset registration.
Hence, on balance we think it is better to have a stateless system, even if it relegates  features like decentralised exchanges to be Layer 2 solutions.

% MPJ: I don't think this adds anything

%Similarly, when we think about facilitating the trading custom tokens, we might want to include or facilitate a
%\emph{decentralised exchange} as part of our system.
%However, again such a listing requires global state to track the offers.

%Note that our model allows for a fair and reliable way for users
%to make offers of MA tokens
%for sale or exchange on the ledger. To make such an offer, a user would place
%their tokens in an output locked by a script. This script requires the transaction spending
%this output to \emph{pay} for the token.

%It is possible to implement a decentralised exchange listing or a global
%asset registry in a
%in a way that does not compromise security guarantees of our current model, but
%at the expense of high overhead, complicated validation rules,
%and frequent transaction validation failure.
%All transactions submitted, but not processed, before a registry update would
%necessarily become invalid, which would be a frequent occurence.

%Even with a design of a registry that solves these problems, it does not make
%sense to have either registry be the main
%source of truth in our model, since their functionality must be specified
%elsewhere, i.e. the relevant forging and output scripts.

\subsubsection{Spending policies.}

Some platforms we discussed provide ways to express restrictions on the \emph{transfer} of
tokens, not just on their forging and burning, which we refer to as \textit{spending policies}.
Unlike forging policies, spending polices are not a native part of our system.
We have considered a number of approaches to adding spending policies, but we
have not found a solution that does not put an undue burden on the \emph{users}
of such tokens, both humans and programmatic users such as layer-two protocols
(e.g. Lightning). For example, it would be necessary to ensure that spending
policies are not in conflict with forging or output-locking scripts (any time the asset is spent).

Forging tokens requires a specific action by the user (providing and satisfying a forging policy script), but this action is always taken knowingly by a user who is specifically trying to forge those tokens.
\emph{Spending} tokens is, however, a completely generic operation that works over arbitrary bundles of tokens.
Indeed, a virtue of our system is that custom tokens all look and behave uniformly.
In contrast, spending policies make custom tokens extremely difficult to handle in a generic way, in particular for automated systems.

These arguments do not invalidate the usefulness of spending policies, but instead highlight that
they are not obviously compatible with trading of native assets in a generic way, and an approach that addresses these issues in an ergonomic way is much needed.
This problem is not ours alone: spending policies in other systems we have looked at here (such as Waves),
do not provide universal solutions to the issues we face with spending policies either.

\subsubsection{Viral scripts.}

One way to emulate spending policies in our system is to lock all the tokens with a particular script that ensures
that they \emph{remain} locked by the same script when transferred.
We call such a script a ``viral'' script (since it spreads to any new outputs that are ``infected'' with the token).

This allows the conditions of the script to be enforced on every transaction that uses the tokens, but at significant costs.
In particular such tokens can never be locked by a \emph{different} script, which prevents such tokens from being used in smart contracts, as well as preventing an output from containing tokens from two such viral asset groups (since \emph{both} would require that their validator be applied to the output!).
In some cases, however, this approach is exactly what we want.
For example, in the case of credential tokens, we want
the script locking the credential to permanently allow the issuer access to the credential (in
order to maintain their ability to revoke it).

\subsubsection{Global State.}

There are limitations of our model due to the fact that global information about the ledger is not available to the forging policy script.
Many global state constraints can be accommodated with workarounds (such as in the case of provably unique fresh tokens), but some cannot:
for example, a forging policy that allows a variable amount to be forged in every block depending on the current total supply of assets of another policy.
This is an odd policy to have, but nevertheless, not one that can be defined in our model.

\subsection{Conclusions}

We present a multi-asset ledger model which extends a UTXO-based ledger so that it can support custom fungible and non-fungible tokens.
We do this in a way that does not require smart contract functionality.
We add a small language with the ability to express exactly the logic we need for a particular set of usecases.
Our \UTXOma{} ledger together with this language allows us to use custom assets to support a wide variety of usecases, even those that are not normally based on ``assets'', while still remaining deterministic, high-assurance, and analysable.

Our design has a number of limitations, some of which have acceptable workarounds, and some that do not.
In particular, access to global state in a general way cannot be supported, and spending policies are not easy to implement.
It is also not possible to explicitly restrict payment to an address.
We consider these worthy directions for future improvements to our model.


% MPJ: This doesn't seem relevant? Nothing to do with multi-asset really
%\paragraph{Transfer restrictions imposed by the recepient.}
%
%In our system, we do not provide a default way to restrict payment to
%an address. We have discussed two platforms where this is possible, Nervos and
%DAML, both of which have fundamentally different designs that compromise other
%features we require in ours. These types of restriction policies
%are closely related to and suffer from many of the same problems as
%spending policies, such as high overhead, clashing constraints with sequetial or
%parallel contracts, and the possibility of circumvention.

% Doesn't tell anything new -=chak
%\section{Conclusion}
%
%We have presented \UTXOma{}, a \UTXO{} ledger model with native, lightweight, stateless mulitasset support.
%\UTXOma{} supports a wide range of key use cases, including fungible and non-fungible tokens.
%
%Moreover, we show that a simple scripting model is adequate to implement a variety of both useful forging policies and output-locking scripts.
%Using these together we are able to realise a wide variety of applications.

% MPJ: I think we can get away with a simpler conclusion

%With our two-level token classification structure, we are able to both
%\begin{itemize}
%  \item infer the fungibility relation directly by looking at the names and IDs of
%  token classes, without executing code
%  \item retain the ability to impose fungibility constraints, as tokens under a single
%  policy are minted and burned, by rejecting attempts to forge tokens that do not satisfy them
%\end{itemize}

%As promised, we gave examples of clauses that a language would need in order to express
%a number of interesting and realistic use cases.
%Defining language clauses in this way, however, is one step away from hard-coding scripts into
%rules about processing transactions, and frequently requires updating them
%every time we encounter a new use case.

%There are many advantages to making this system more expressive, with a fully
%formed scripting language. The approach we present here, however, gives use the opportunity to highlight
%just how versatile and intuitive our MA structure us due to being native and
%represented by the two-level map we described. Even with a minimally expressive
%smart contract language (only checking multi-signatures), we can already allow users to
%define asset policies governing the forging of custom tokens.
