\section{A stateless forging policy language}
\label{sec:fps-language}

The domain-specific language for forging policies strikes a balance between expressiveness and simplicity. It particular, it is stateless and of bounded computational complexity. Nevertheless, it is sufficient to support the applications described in Section~\ref{sec:applications}.

\paragraph{Semantically meaningful token names.}

The policy ID is associated with a policy script (it is the hash of it),
so it has a semantic meaning that is identified with that of the script.
In the clauses of our language, we give semantic meaning
to the names of the tokens as well. This allows us to make some
judgements about them in a programmatic way, beyond confirming that the
preservation of value holds, or
which ones are fungible with each other.
For example, the $\texttt{FreshTokens}$ constructor gives us a way to programmatically generate
token names which, by construction, mean that these tokens are unique, without
ever checking the global ledger state.

\paragraph{Forging policy scripts as output-locking scripts.}

As with
currency in the non-digital world, it is a harder problem to control the transfer of assets once
they have come into circulation (see also Section~\ref{sec:discussion}). We can, however,
specify directly in the forging policy that the assets being forged must
be locked by an output script of our choosing.
Moreover, since both output addresses and policies are hashes of scripts,
we can use the asset policy ID and the address interchangeably.
The $\texttt{AssetToAddress}$ clause is used for this purpose.

\begin{figure}[t]
    \begin{lstlisting}
    $\llbracket$JustMSig(msig)$\rrbracket$(h, tx, utxo) = checkMultiSig(msig, tx)

    $\llbracket$SpendsOutput(o)$\rrbracket$(h, tx, utxo) = o $\in $ { i.outputRef : i $\in $ tx.inputs }

    $\llbracket$TickAfter(tick1)$\rrbracket$(h, tx, utxo) = tick1 $\leq$ min(tx.validityInterval)

    $\llbracket$Forges(tkns)$\rrbracket$(h, tx, utxo) = (h $\mapsto$ tkns $\in$ tx.forge) && (h $\mapsto$ tkns $\geq$ 0)

    $\llbracket$Burns(tkns)$\rrbracket$(h, tx, utxo) = (h $\mapsto$ tkns $\in$ tx.forge) && (h $\mapsto$ tkns $\leq$ 0)

    $\llbracket$FreshTokens$\rrbracket$(h, tx, utxo) =
       $\forall$ pid $\mapsto$ tkns $\in$ tx.forge, pid == h $\Rightarrow$
         $\forall$ t $\mapsto$ q $\in$ tkns,
           t == hash(indexof(t, tkns), tx.inputs) && q == 1

    $\llbracket$AssetToAddress(addr)$\rrbracket$(h, tx, utxo) =
       $\forall$ pid $\mapsto$ tkns $\in$ utxo.balance, pid == h $\Rightarrow$
          addr == _ $\Rightarrow$ (h, pid $\mapsto$ tkns) $\in$ utxo
        $\wedge$ addr $\neq$ _ $\Rightarrow$ (addr, pid $\mapsto$ tkns) $\in$ utxo

    $\llbracket$DoForge$\rrbracket$(h, tx, utxo) = h $\in$ supp(tx.forge)

    $\llbracket$SignedByPIDToken$\rrbracket$(h, tx, utxo) =
       $\forall$ pid $\mapsto$ tkns $\in$ utxo.balance, pid == h $\Rightarrow$
         $\forall$ s $\in$ tx.sigs, $\exists$ t $\in$ supp(tkns),
           isSignedBy(tx, s, t)

    $\llbracket$SpendsCur(pid)$\rrbracket$(h, tx, utxo) =
         pid == _ $\Rightarrow$ h $\in$ supp(utxo.balance)
       $\wedge$ pid $\neq$ _ $\Rightarrow$ pid $\in$ supp(utxo.balance)
    \end{lstlisting}
    \caption{Forging Policy Language}
    \label{figure:fps-language}
\end{figure}
%
\paragraph{Language clauses.}
%
The various clauses of the validator and forging policy language are as described below, with their formal semantics as in Figure~\ref{figure:fps-language}. In this figure, we use the notation
$\texttt{x}\mapsto \texttt{y}$ to represent a single key-value pair of a finite map.
Recall form Rule~\ref{rule:all-mps-validate} that the arguments passed to the validation
function $\llbracket \texttt{s} \rrbracket$ $\texttt{h}$ are: the hash of the forging (or output
locking) script being validated, the transaction $\texttt{tx}$ being validated,
and the ledger outputs which the transaction $\texttt{tx}$ is spending (we denote
these $\texttt{utxo}$ here).
%
\begin{itemize}
  \item \texttt{JustMSig(msig)}
  verifies that the $m$-out-of-$n$ signatures required by $s$ are in the
  set of signatures provided by the transaction. We do not give the
  multi-signature script evaluator details as this is a common concept, and
  assume a procedure \texttt{checkMultiSig} exists.

  \item \texttt{SpendsOutput(o)} checks that the transaction spends
  the output referenced by $\texttt{o}$ in the \UTXO.

  \item \texttt{TickAfter(tick1)} checks that the validity interval
  of the current transaction starts after time $\texttt{tick1}$.

  \item \texttt{Forges(tkns)} checks that the transaction forges exactly
  $\texttt{tkns}$ of the asset with the policy ID that is being validated.

  \item \texttt{Burns(tkns)} checks that the transaction burns exactly
  $\texttt{tkns}$ of the asset with the policy ID that is being validated.

  \item \texttt{FreshTokens}
  checks that all tokens of the asset being forged are non-fungible.

  This script must check that the names of the tokens in this token bundle
  are generated by hashing some unique data. This data must be unique to both
  the transaction itself and the token within the asset being forged.
  In particular, we can hash a pair of
  %
  \begin{enumerate}
    \item some \emph{output} in the \UTXO\ that the
    transaction consumes, and
    \item the \emph{index} of the token name in (the
    list representation of) the map of tokens being
    forged (under the specific policy, by this transaction). We denote the
    function that gets the index of a key in a key-value map by $\texttt{indexof}$.
  \end{enumerate}

  \item \texttt{AssetToAddress(addr)}
  checks that all the tokens associated with the policy ID that
  is equal to the hash of the script being run are output
  to an \UTXO\ with the address $\texttt{addr}$. In the case that no $\texttt{addr}$ value is
  provided (represented by $\_$), we use the $\texttt{addr}$ value passed to the evaluator as
  the hash of the policy of the asset being forged.

  \item \texttt{DoForge}
  checks that this transaction forges tokens in the bundle controlled by the policy ID that is passed
  to the FPS script evaluator (here, again, we make use of the separate passing of
  the FPS script and the policy ID).

  \item \texttt{SignedByPIDToken(pid)} verifies the hash of every key that has signed the transaction.

  \item \texttt{SpendsCur(pid)}
  verifies that the transaction is spending assets in the token bundle with policy
  ID $\texttt{pid}$ (which is specified as part of the \emph{constructor}, and may be different
  than the policy ID passed to the evaluator).

\end{itemize}
