\section{Introduction}

Bitcoin, the most widely known and most valuable cryptocurrency, uses
a graph-based ledger model built on the concept of \emph{\UTXO{}s} (\emph{unspent
  transaction outputs})~\cite{formal-model-of-bitcoin-transactions,Zahnentferner18-UTxO}. Individual \emph{transactions} consist of a list of \emph{inputs} and a list of \emph{outputs}, where outputs represent a specific \emph{value} (of a cryptocurrency) that is available to be spent by inputs of subsequent transactions. Each output can be spent by (i.e., connect to) exactly one input. Moreover, we don't admit cycles in these connections, and hence we can regard a collection of transactions spending from each other as a directed acyclic graph, where a transaction with $m$ inputs and $n$ outputs is represented by a node in the graph with $m$ edges in and $n$ edges out.
The sum of the values consumed by a transaction's inputs must equal the sum of the values provided by its outputs, thus value is conserved.

Whether an output can be consumed by an input is determined by a function $\nu$ attached to the output, which we call the output's \emph{validator}. A transaction input proves its eligibility to spent an output by providing a \emph{redeemer} object $\rho$, such that \(\nu(\rho) = \true\); redeemers are often called \emph{witnesses} in Bitcoin. In the simplest case, the redeemer is a cryptographic hash of the spending transaction signed by an authorised spender's private key, which is checked by the validator, which embeds the corresponding public key. More sophisticated protocols are possible by using more complex validator functions and redeemers --- see \cite{bitml} for a high-level model of what is possible with the functionality provided by Bitcoin.

The benefit of this graph-based approach to a cryptocurrency ledger is that it plays well with the concurrent and distributed nature of blockchains. In particular, it forgoes any notion of shared mutable state, which is known to lead to highly complex semantics in the face of concurrent and distributed computations involving that shared mutable state.

Nevertheless, the \UTXO{} model, generally, and Bitcoin, specifically, has been criticised for the limited expressiveness of programmability achieved by the validator concept. In particular, Ethereum's \emph{account-based ledger} and the associated notion of \emph{contract accounts} has been motivated by the desire to overcome those limitations. Unfortunately, it does so by introducing a notion of shared mutable state, which significantly complicates the semantics of contract code. In particular, contract authors need to understand the subtleties of this semantics or risk introducing security issues (such as the vulnerability to recursive contract invocations that led to the infamous DAO attack~\cite{DAO-attack}).

\paragraph{Contributions.}

The contribution of the present paper is to propose an extension to the basic \UTXO{} ledger model, which
\begin{inparaenum}[(a)]
\item provably increases expressiveness, while simultaneously
\item preserving the dataflow properties of the \UTXO{} graph; in particular, it forgoes introducing any notion of shared mutable state.
\end{inparaenum}
More specifically, we make the following contributions:
%
\begin{itemize}
\item We propose the \emph{\EUTXO{} model}, informally in Section~\ref{sec:informal-eutxo} and formally in Section~\ref{sec:formal-model}.
\item We demonstrate that the \EUTXO{} model supports the implementation of a specific form of state machines (\emph{Constraint Emitting Machines}, or \CEM{}s), which the basic \UTXO{} model does not support, in Section~\ref{sec:expressiveness}.
\item We provide formalisations of both the \EUTXO{} model
  and Constraint Emitting Machines. We prove a weak bisimulation
  between the two using the Agda proof
  assistant\site{https://github.com/\GitUser/formal-utxo/tree/\AgdaCommit}, building on
  previous work by Melkonian et al.~\cite{formal-eutxo}.
\end{itemize}

\noindent Section~\ref{sec:related} summarises related work.

The \EUTXO{} model will be used in the ledger of \Cardano{}, a major blockchain
system currently being developed by IOHK.
It also provides the foundation of Cardano's smart contract platform
\emph{Plutus}\site{https://github.com/input-output-hk/plutus}, which includes a small
functional programming language \emph{\Plutus{}} which is used to implement \script{}s.
Although a technical description of Cardano itself is beyond the scope of this paper,
one can try out the Plutus Platform in an online playground.\site{https://prod.playground.plutus.iohkdev.io/}

Other future work includes a formal comparison of \EUTXO{} with Ethereum's account-based model.

\iftoggle{anonymous}{
\paragraph{Anonymisation.}

For the purposes of submission, we have anonymised the following names:
\begin{itemize}
\item \Cardano{}: a major blockchain system.
\item \Plutus{}: a small functional programming language.
\end{itemize}
}{}
